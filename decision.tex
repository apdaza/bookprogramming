\chapter{Estructura de Decisi�n}\label{cap:decision}

\section{Estructura Si-No}

%=======================================================================================
%=======================================================================================
\paragraph*{\ejercicio}  \par %ejercicio20
Escribir un programa que detecte si un n�mero introducido desde el teclado es positivo o negativo.

%Lenguaje C
\lstinputlisting[language=C, caption={Lenguaje C $\Longleftarrow$ Ejercicio \theejercicio}, label={\theejercicio b}]{ejercicios/c++/ejercicio020.cpp}


%lenguaje python
\lstinputlisting[language=Python, caption={Lenguaje Python $\Longleftarrow$ Ejercicio \theejercicio}, label={\theejercicio d}]{ejercicios/python/ejercicio020.py}

%=======================================================================================
%=======================================================================================


%=======================================================================================
%=======================================================================================
\paragraph*{\ejercicio}  \par %ejercicio21
Escribir un programa que determine si un n�mero le�do desde el teclado es par o impar.\\

%Lenguaje C
\lstinputlisting[language=C, caption={Lenguaje C $\Longleftarrow$ Ejercicio \theejercicio}, label={\theejercicio b}]{ejercicios/c++/ejercicio021.cpp}


%lenguaje python
\lstinputlisting[language=Python, caption={Lenguaje Python $\Longleftarrow$ Ejercicio \theejercicio}, label={\theejercicio d}]{ejercicios/python/ejercicio021.py}

%=======================================================================================
%=======================================================================================

%=======================================================================================
%=======================================================================================
\paragraph*{\ejercicio}  \par %ejercicio 022
Escribir un programa que una vez le�da una hora en formato (horas, minutos, segundos) indique cual ser� el tiempo dentro de un segundo

%Lenguaje C
\lstinputlisting[language=C, caption={Lenguaje C $\Longleftarrow$ Ejercicio \theejercicio}, label={\theejercicio b}]{ejercicios/c++/ejercicio022.cpp}

%lenguaje python
\lstinputlisting[language=Python, caption={Lenguaje Python $\Longleftarrow$ Ejercicio \theejercicio}, label={\theejercicio d}]{ejercicios/python/ejercicio022.py}

%=======================================================================================
%=======================================================================================

%=======================================================================================
%=======================================================================================
\paragraph*{\ejercicio}  \par %ejercicio23
Escribir un programa que dados dos n�meros enteros positivos y una vez realizadas las comprobaciones necesarias muestre por pantalla uno de los siguientes mensajes:
\begin{itemize}
\item El segundo es el cuadrado exacto del primero.
\item El segundo es menor que el cuadrado del primero.
\item El segundo es mayor que el cuadrado del primero.
\end{itemize}

%Lenguaje C
\lstinputlisting[language=C, caption={Lenguaje C $\Longleftarrow$ Ejercicio \theejercicio}, label={\theejercicio b}]{ejercicios/c++/ejercicio023.cpp}

%lenguaje python
\lstinputlisting[language=Python, caption={Lenguaje Python $\Longleftarrow$ Ejercicio \theejercicio}, label={\theejercicio d}]{ejercicios/python/ejercicio023.py}
%=======================================================================================
%=======================================================================================

%=======================================================================================
%=======================================================================================
\paragraph*{\ejercicio}  \par %ejercicio 024
Escribir un programa que calcula el equivalente en grados Fahrenheit o Celsius de una temperatura t, el usuario debe indicar si la temperatura que ingreso esta en celcius o fahrenheit acompa�ando el valor por el car�cter c o t respectivamente. (Celsius / 5 = (Fahrenheit - 32) * 9)

%Lenguaje C
\lstinputlisting[language=C, caption={Lenguaje C $\Longleftarrow$ Ejercicio \theejercicio}, label={\theejercicio b}]{ejercicios/c++/ejercicio024.cpp}

%lenguaje python
\lstinputlisting[language=Python, caption={Lenguaje Python $\Longleftarrow$ Ejercicio \theejercicio}, label={\theejercicio d}]{ejercicios/python/ejercicio024.py}

%=======================================================================================
%=======================================================================================

%=======================================================================================
%=======================================================================================
\paragraph*{\ejercicio}  \par %ejercicio 025
Escrbir un programa que dado un n�mero entre 1 y 12 indique cuantos d�as tiene el mes corespondiente

%Lenguaje C
\lstinputlisting[language=C, caption={Lenguaje C $\Longleftarrow$ Ejercicio \theejercicio}, label={\theejercicio b}]{ejercicios/c++/ejercicio025.cpp}

%lenguaje python
\lstinputlisting[language=Python, caption={Lenguaje Python $\Longleftarrow$ Ejercicio \theejercicio}, label={\theejercicio d}]{ejercicios/python/ejercicio025.py}
%=======================================================================================
%=======================================================================================

%=======================================================================================
%=======================================================================================
\paragraph*{\ejercicio}  \par %ejercicio 026
Escribir un programa que determine si un a�o es bisiesto. Un a�o es bisiesto si es m�ltiplo de 4. Los a�os m�ltiplos de 100 no son bisiestos, salvo si ellos son tambi�n m�ltiplos de 400. 

%Lenguaje C
\lstinputlisting[language=C, caption={Lenguaje C $\Longleftarrow$ Ejercicio \theejercicio}, label={\theejercicio b}]{ejercicios/c++/ejercicio026.cpp}

%lenguaje python
\lstinputlisting[language=Python, caption={Lenguaje Python $\Longleftarrow$ Ejercicio \theejercicio}, label={\theejercicio d}]{ejercicios/python/ejercicio026.py}
%=======================================================================================
%=======================================================================================

%=======================================================================================
%=======================================================================================
\paragraph*{\ejercicio}  \par %ejercicio 027
Escribir un programa que capture un a�o, un mes y un d�a e indique si es una fecha v�lida o no

%Lenguaje C
\lstinputlisting[language=C, caption={Lenguaje C $\Longleftarrow$ Ejercicio \theejercicio}, label={\theejercicio b}]{ejercicios/c++/ejercicio027.cpp}

%lenguaje python
\lstinputlisting[language=Python, caption={Lenguaje Python $\Longleftarrow$ Ejercicio \theejercicio}, label={\theejercicio d}]{ejercicios/python/ejercicio027.py}
%=======================================================================================
%=======================================================================================

%=======================================================================================
%=======================================================================================
\paragraph*{\ejercicio}  \par %ejercicio 028
Escribir un programa que le indique al usuario si su peso es adecuado de acuerdo a su estatura, teniendo en cuenta la f�rmula de Hamwi: cuando el sexo es femenino es: 45.5 kg. + 0.866 * [altura (cm) - 152.4]; si el sexo es masculino es:  50 kg. + 1.06 * [altura (cm) - 152.4].

%Lenguaje C
\lstinputlisting[language=C, caption={Lenguaje C $\Longleftarrow$ Ejercicio \theejercicio}, label={\theejercicio b}]{ejercicios/c++/ejercicio028.cpp}

%lenguaje python
\lstinputlisting[language=Python, caption={Lenguaje Python $\Longleftarrow$ Ejercicio \theejercicio}, label={\theejercicio d}]{ejercicios/python/ejercicio028.py}
%=======================================================================================
%=======================================================================================

%=======================================================================================
%=======================================================================================
\paragraph*{\ejercicio}  \par %ejercicio 029
Escribir un programa que despues de leer 3 n�meros retorne el n�mero medio de los tres. El n�mero medio es aquel que no es el mayor ni el menor de los 3 n�meros
\lstinputlisting[language=C, caption={Lenguaje C $\Longleftarrow$ Ejercicio \theejercicio}, label={\theejercicio b}]{ejercicios/c++/ejercicio028.cpp}

%lenguaje python
\lstinputlisting[language=Python, caption={Lenguaje Python $\Longleftarrow$ Ejercicio \theejercicio}, label={\theejercicio d}]{ejercicios/python/ejercicio028.py}
%=======================================================================================
%=======================================================================================


\section{Estructura de selecci�n}

%=======================================================================================
%=======================================================================================
\paragraph*{\ejercicio}  \par %ejercicio 030
escribir un programa que dado un n�mero del 1 al 12 indique a que mes corresponde y cuantos d�as tiene el mes.
%Lenguaje C
\lstinputlisting[language=C, caption={Lenguaje C $\Longleftarrow$ Ejercicio \theejercicio}, label={\theejercicio b}]{ejercicios/c++/ejercicio030.cpp}

%lenguaje python
\lstinputlisting[language=Python, caption={Lenguaje Python $\Longleftarrow$ Ejercicio \theejercicio}, label={\theejercicio d}]{ejercicios/python/ejercicio030.py}
%=======================================================================================
%=======================================================================================


%=======================================================================================
%=======================================================================================
\paragraph*{\ejercicio}  \par %ejercicio31
Escribir un programa que dado un n�mero del 1 a 7 escriba el correspondiente nombre del d�a de la semana. 

%Lenguaje C
\lstinputlisting[language=C, caption={Lenguaje C $\Longleftarrow$ Ejercicio \theejercicio}, label={\theejercicio b}]{ejercicios/c++/ejercicio031.cpp}

%lenguaje python
\lstinputlisting[language=Python, caption={Lenguaje Python $\Longleftarrow$ Ejercicio \theejercicio}, label={\theejercicio d}]{ejercicios/python/ejercicio031.py}

%=======================================================================================
%=======================================================================================

%=======================================================================================
%=======================================================================================
\paragraph*{\ejercicio}  \par %ejercicio32
Escribir un programa que lea un car�cter e indique si es o no una vocal.\\

%Lenguaje C
\lstinputlisting[language=C, caption={Lenguaje C $\Longleftarrow$ Ejercicio \theejercicio}, label={\theejercicio b}]{ejercicios/c++/ejercicio032.cpp}

%lenguaje python
\lstinputlisting[language=Python, caption={Lenguaje Python $\Longleftarrow$ Ejercicio \theejercicio}, label={\theejercicio d}]{ejercicios/python/ejercicio032.py}
%=======================================================================================
%=======================================================================================

%=======================================================================================
%=======================================================================================
\paragraph*{\ejercicio}  \par %ejercicio33
Escribir un prgrama que permita al usuario elegir el c�lculo del �rea de cualquiera de las figuras geom�tricas: c�rculo, triangulo y rect�ngulo, usando las formulas PI * r2 (circulo), (b * h)/2 (triangulo), b * h (rect�ngulo). El progradebe presentar un men� al usuario con las siguientes opciones:
1 - circulo
2 - triangulo
3 - rectangulo

%Lenguaje C
\lstinputlisting[language=C, caption={Lenguaje C $\Longleftarrow$ Ejercicio \theejercicio}, label={\theejercicio b}]{ejercicios/c++/ejercicio033.cpp}

%lenguaje python
\lstinputlisting[language=Python, caption={Lenguaje Python $\Longleftarrow$ Ejercicio \theejercicio}, label={\theejercicio d}]{ejercicios/python/ejercicio033.py}
%=======================================================================================
%=======================================================================================

%=======================================================================================
%=======================================================================================
\paragraph*{\ejercicio}  \par %ejercicio34
Escribir un programa que dado un car�cter num�rico devuelva su correspondiente n�mero.
%Lenguaje C
\lstinputlisting[language=C, caption={Lenguaje C $\Longleftarrow$ Ejercicio \theejercicio}, label={\theejercicio b}]{ejercicios/c++/ejercicio034.cpp}

%lenguaje python
\lstinputlisting[language=Python, caption={Lenguaje Python $\Longleftarrow$ Ejercicio \theejercicio}, label={\theejercicio d}]{ejercicios/python/ejercicio034.py}
%=======================================================================================
%=======================================================================================


