\chapter{Estructuras de Datos}\label{cap:estructurasdatos}

\section{Arreglos}

%=======================================================================================
%=======================================================================================
\paragraph*{\ejercicio}  \par %Ejercicio 078
Escribir un programa que dado un arreglo de las vocales y otro con las consonantes, capture una cadena de caracteres y cuente cuantos caracteres pertenecen al arreglo de vocales, cuantos al arreglo de consonantes y cuantos no pertenecen a ninguno de los dos.

%Lenguaje C
\lstinputlisting[language=C, caption={Lenguaje C $\Longleftarrow$ Ejercicio \theejercicio}, label={\theejercicio b}]{ejercicios/c++/ejercicio078.cpp}

%lenguaje python
\lstinputlisting[language=Python, caption={Lenguaje Python $\Longleftarrow$ Ejercicio \theejercicio}, label={\theejercicio d}]{ejercicios/python/ejercicio078.py}
%=======================================================================================
%=======================================================================================

%=======================================================================================
%=======================================================================================
\paragraph*{\ejercicio}  \par %Ejercicio 079
Escribir un programa que usando arreglos de d�gitos validos verifique si un n�mero entero positivo ingresado por el usuario es v�lido en un sistema con base X (2,8,10) dada tambi�n por el usuario.

%Lenguaje C
\lstinputlisting[language=C, caption={Lenguaje C $\Longleftarrow$ Ejercicio \theejercicio}, label={\theejercicio b}]{ejercicios/c++/ejercicio079.cpp}

%lenguaje python
\lstinputlisting[language=Python, caption={Lenguaje Python $\Longleftarrow$ Ejercicio \theejercicio}, label={\theejercicio d}]{ejercicios/python/ejercicio079.py}
%=======================================================================================
%=======================================================================================

%=======================================================================================
%=======================================================================================
\paragraph*{\ejercicio}  \par %ejercicio 080
Escribir un programa que tras asignar n�meros negativos y positivos de manera aleatoria a un arreglo de 10 posiciones calcule, independientemente, la suma de los elementos positivos y negativos.

%Lenguaje C
\lstinputlisting[language=C, caption={Lenguaje C $\Longleftarrow$ Ejercicio \theejercicio}, label={\theejercicio b}]{ejercicios/c++/ejercicio080.cpp}

%lenguaje python
\lstinputlisting[language=Python, caption={Lenguaje Python $\Longleftarrow$ Ejercicio \theejercicio}, label={\theejercicio d}]{ejercicios/python/ejercicio080.py}
%=======================================================================================
%=======================================================================================

%=======================================================================================
%=======================================================================================
\paragraph*{\ejercicio}  \par %ejercicio 081
Escribir un programa que tras asignar 10 n�meros de manera aleatoria a un arreglo, determine las posiciones de este en las que se encuentran el m�ximo, el m�nimo valor y el valor m�s cercano al promedio entero. \\

%Lenguaje C
\lstinputlisting[language=C, caption={Lenguaje C $\Longleftarrow$ Ejercicio \theejercicio}, label={\theejercicio b}]{ejercicios/c++/ejercicio081.cpp}

%lenguaje python
\lstinputlisting[language=Python, caption={Lenguaje Python $\Longleftarrow$ Ejercicio \theejercicio}, label={\theejercicio d}]{ejercicios/python/ejercicio081.py}

%=======================================================================================
%=======================================================================================

%=======================================================================================
%=======================================================================================
\paragraph*{\ejercicio}  \par %ejercicio 082
Escribir un programa que inicialice un arreglo con n�meros aleatorios enteros entre $0$ y $300$ y calcule por medio de un m�todo recursivo el elemento mayor.

%Lenguaje C
\lstinputlisting[language=C, caption={Lenguaje C $\Longleftarrow$ Ejercicio \theejercicio}, label={\theejercicio b}]{ejercicios/c++/ejercicio082.cpp}

%lenguaje python
\lstinputlisting[language=Python, caption={Lenguaje Python $\Longleftarrow$ Ejercicio \theejercicio}, label={\theejercicio d}]{ejercicios/python/ejercicio082.py}
%=======================================================================================
%=======================================================================================


%=======================================================================================
%=======================================================================================
\paragraph*{\ejercicio}  \par %ejercicio 083
Escribir un programa que usando un arreglos de enteros de 10 posiciones, inicializado con numeros aleatorios entre $100$ y $200$ permita ordenar dicho arreglo usando un m�todo de ordenamiento de burbuja.

%Lenguaje C
\lstinputlisting[language=C, caption={Lenguaje C $\Longleftarrow$ Ejercicio \theejercicio}, label={\theejercicio b}]{ejercicios/c++/ejercicio083.cpp}

%lenguaje python
\lstinputlisting[language=Python, caption={Lenguaje Python $\Longleftarrow$ Ejercicio \theejercicio}, label={\theejercicio d}]{ejercicios/python/ejercicio083.py}
%=======================================================================================
%=======================================================================================


%=======================================================================================
%=======================================================================================



\section{Matrices}

%=======================================================================================
%=======================================================================================
\paragraph*{\ejercicio}  \par %ejercicio 084
Escribir un programa que determine la posici�n de una matriz de $3 x 3$ en la que se encuentra el valor m�ximo de una serie de n�meros generados aleatoriamente.

%Lenguaje C
\lstinputlisting[language=C, caption={Lenguaje C $\Longleftarrow$ Ejercicio \theejercicio}, label={\theejercicio b}]{ejercicios/c++/ejercicio084.cpp}

%lenguaje python
\lstinputlisting[language=Python, caption={Lenguaje Python $\Longleftarrow$ Ejercicio \theejercicio}, label={\theejercicio d}]{ejercicios/python/ejercicio084.py}
%=======================================================================================
%=======================================================================================

%=======================================================================================
%=======================================================================================
\paragraph*{\ejercicio}  \par %ejercicio 085
Escribir un programa que sume, independientemente, los elementos positivos y negativos de la una matriz de $5 x 5$, la matriz se debe llenar solicitando los datos al usuario.

%Lenguaje C
\lstinputlisting[language=C, caption={Lenguaje C $\Longleftarrow$ Ejercicio \theejercicio}, label={\theejercicio b}]{ejercicios/c++/ejercicio085.cpp}

%lenguaje python
\lstinputlisting[language=Python, caption={Lenguaje Python $\Longleftarrow$ Ejercicio \theejercicio}, label={\theejercicio d}]{ejercicios/python/ejercicio085.py}
%=======================================================================================
%=======================================================================================

%=======================================================================================
%=======================================================================================
\paragraph*{\ejercicio}  \par %ejercicio 086
Escribir un programa que llene la primera fila de una matriz de 3 filas por 10 columnas con n�meros aleatorios entre 1 y 20, la segunda fila con los cuadrados de los datos de la primera fila y la tercera fila con la suma de la primera y la segunda.

%Lenguaje C
\lstinputlisting[language=C, caption={Lenguaje C $\Longleftarrow$ Ejercicio \theejercicio}, label={\theejercicio b}]{ejercicios/c++/ejercicio086.cpp}

%lenguaje python
\lstinputlisting[language=Python, caption={Lenguaje Python $\Longleftarrow$ Ejercicio \theejercicio}, label={\theejercicio d}]{ejercicios/python/ejercicio086.py}
%=======================================================================================
%=======================================================================================

%=======================================================================================
%=======================================================================================
\paragraph*{\ejercicio}  \par %ejercicio 087
Escribir un programa que divida todos los elementos de una matriz $M (3 , 4)$ por el elemento situado en la posici�n $(2,2)$ .

%Lenguaje C
\lstinputlisting[language=C, caption={Lenguaje C $\Longleftarrow$ Ejercicio \theejercicio}, label={\theejercicio b}]{ejercicios/c++/ejercicio087.cpp}

%lenguaje python
\lstinputlisting[language=Python, caption={Lenguaje Python $\Longleftarrow$ Ejercicio \theejercicio}, label={\theejercicio d}]{ejercicios/python/ejercicio087.py}
%=======================================================================================
%=======================================================================================

%=======================================================================================
%=======================================================================================
\paragraph*{\ejercicio}  \par %ejercicio 088
Escribir un programa que permita ordenar los elementos de una matriz de $M (5, 5) $ usando el m�todo de burbuja, el programa debe inicializar la matriz con n�meros aleatorios multiplos de 5 comprendidos entre $1$ y $100$ y mostrar la matriz en pantalla de forma ordenada tabulando a derecha cada uno de los elementos de la matriz.

%Lenguaje C
\lstinputlisting[language=C, caption={Lenguaje C $\Longleftarrow$ Ejercicio \theejercicio}, label={\theejercicio b}]{ejercicios/c++/ejercicio088.cpp}

%lenguaje python
\lstinputlisting[language=Python, caption={Lenguaje Python $\Longleftarrow$ Ejercicio \theejercicio}, label={\theejercicio d}]{ejercicios/python/ejercicio088.py}
%=======================================================================================
%=======================================================================================


%=======================================================================================
%=======================================================================================

%=======================================================================================
%=======================================================================================
\paragraph*{\ejercicio}  \par %ejercicio 089
Dada una matriz cuadrada de $4$ por $4$ inicializada con n�meros aleatorios comprendidos entre $10$ y $50$, escribir un programa que genere una matriz que al sumarla elemento a elemento con la matriz original convierta en $1$ los n�meros de la matriz del triangulo superior a la diagonal principal de la matriz y en $-1$ los n�meros del triangulo inferior a la diagonal principal, los n�meros de la diagonal principal deben convertirce a $0$. Se debe mostrar en pantalla la matriz original, la matriz generada para convertir la original y la matriz convertida.

%Lenguaje C
\lstinputlisting[language=C, caption={Lenguaje C $\Longleftarrow$ Ejercicio \theejercicio}, label={\theejercicio b}]{ejercicios/c++/ejercicio089.cpp}

%lenguaje python
\lstinputlisting[language=Python, caption={Lenguaje Python $\Longleftarrow$ Ejercicio \theejercicio}, label={\theejercicio d}]{ejercicios/python/ejercicio089.py}
%=======================================================================================
%=======================================================================================

%=======================================================================================
%=======================================================================================


%=======================================================================================
%=======================================================================================
\paragraph*{\ejercicio}  \par %ejercicio 090
Escribir un programa que inicialice una matriz de $M(5, 5)$ con n�meros aleatorios entre $1$ y $999$ y que genere otra matriz $Z(5, 5)$ que contenga en cada celda el conteo de los digitos de cada n�mero en la matriz $M$.

%Lenguaje C
\lstinputlisting[language=C, caption={Lenguaje C $\Longleftarrow$ Ejercicio \theejercicio}, label={\theejercicio b}]{ejercicios/c++/ejercicio090.cpp}

%lenguaje python
\lstinputlisting[language=Python, caption={Lenguaje Python $\Longleftarrow$ Ejercicio \theejercicio}, label={\theejercicio d}]{ejercicios/python/ejercicio090.py}
%=======================================================================================
%=======================================================================================

%=======================================================================================
%=======================================================================================


% Linear Cellular Automata 

%A biologist is experimenting with DNA modification of bacterial colonies being grown in a linear array of culture dishes. By changing the DNA, he is able ``program" the bacteria to respond to the population density of the neighboring dishes. Population is measured on a four point scale (from 0 to 3). The DNA information is represented as an array DNA, indexed from 0 to 9, of population density values and is interpreted as follows:

%    In any given culture dish, let K be the sum of that culture dish's density and the densities of the dish immediately to the left and the dish immediately to the right. Then, by the next day, that dish will have a population density of DNA[K].
  %  The dish at the far left of the line is considered to have a left neighbor with population density 0.
    %The dish at the far right of the line is considered to have a right neighbor with population density 0.

%Now, clearly, some DNA programs cause all the bacteria to die off (e.g., [0,0,0,0,0,0,0,0,0,0]). Others result in immediate population explosions (e.g., [3,3,3,3,3,3,3,3,3,3]). The biologist is interested in how some of the less obvious intermediate DNA programs might behave.

%Write a program to simulate the culture growth in a line of 40 dishes, assuming that dish 20 starts with a population density of 1 and all other dishes start with a population density of 0.

%Input

%The input begins with a single positive integer on a line by itself indicating the number of the cases following, each of them as described below. This line is followed by a blank line, and there is also a blank line between two consecutive inputs.

%For each input set your program will read in the DNA program (10 integer values) on one line.

%Output

%For each test case, the output must follow the description below. The outputs of two consecutive cases will be separated by a blank line.

%For each input set it should print the densities of the 40 dishes for each of the next 50 days. Each day's printout should occupy one line of 40 characters. Each dish is represented by a single character on that line. Zero population densities are to be printed as the character ` '. Population density 1 will be printed as the character `.'. Population density 2 will be printed as the character `x'. Population density 3 will be printed as the character `W'.

%Sample Input

%1

%0 1 2 0 1 3 3 2 3 0

%Sample Output

%bbbbbbbbbbbbbbbbbbb.bbbbbbbbbbbbbbbbbbbb
%bbbbbbbbbbbbbbbbbb...bbbbbbbbbbbbbbbbbbb
%bbbbbbbbbbbbbbbbb.xbx.bbbbbbbbbbbbbbbbbb
%bbbbbbbbbbbbbbbb.bb.bb.bbbbbbbbbbbbbbbbb
%bbbbbbbbbbbbbbb.........bbbbbbbbbbbbbbbb
%bbbbbbbbbbbbbb.xbbbbbbbx.bbbbbbbbbbbbbbb
%bbbbbbbbbbbbb.bbxbbbbbxbb.bbbbbbbbbbbbbb
%bbbbbbbbbbbb...xxxbbbxxx...bbbbbbbbbbbbb
%bbbbbbbbbbb.xb.WW.xbx.WW.bx.bbbbbbbbbbbb
%bbbbbbbbbb.bbb.xxWb.bWxx.bbb.bbbbbbbbbbb

 

%Note: Whe show only the first ten lines of output (the total number of lines must be 50) and the spaces have been replaced with the character "b" for ease of reading. The actual output file will use the ASCII-space character, not "b". 
%http://uva.onlinejudge.org/index.php?option=com_onlinejudge&Itemid=8&category=94&page=show_problem&problem=398
%=======================================================================================
%=======================================================================================

%Context

%This year is the XXV Anniversary of the Faculty of Computer Science in Murcia. But, what XXV means? Maybe you should ask an ancient Roman to get the answer.
%The Problem

%A Roman numeral consists of a set of letters of the alphabet. Each letter has a particular value, as shown in the following table:
%Letter 	I 	V 	X 	L 	C 	D 	M
%Value 	1 	5 	10 	50 	100 	500 	1000

%Generally, Roman numerals are written in descending order from left to right, and are added sequentially. However, certain combinations employ a subtractive principle. If a symbol of smaller value precedes a symbol of larger value, the smaller value is subtracted from the larger value, and the result is added to the total. This subtractive principle follows the next rules:

    %"I" may only precede "V" and "X" (e.g., IV=4).
    %"X" may only precede "L" and "C" (e.g., XC=900).
    %"C" may only precede "D" and "M".
    %"V", "L" and "D" are always followed by a symbol of smaller value, so they are always added to the total.

%Symbols "I", "X", "C" and "M" cannot appear more than three consecutive times. Symbols "V", "L" and "D" cannot appear more than once consecutively.

%Roman numerals do not include the number zero, and for values greater or equal than 4000 they used bars placed above the letters to indicate multiplication by 1000.

%You have write a program that converts from Roman to Arabic numerals and vice versa. Although lower case letters were used in the Middle Ages, the Romans only used upper case letters. Therefore, for the Roman numerals we only consider upper case letters.
%The Input

%The input consists of several lines, each one containing either an Arabic or a Roman number n, where 0 < n < 4000.
%The Output

%For each input line, you must print a line with the converted number. If the number is Arabic, you must give it in Roman format. If the number is Roman, you must give it in Arabic format.
%Sample Input

%XXV
%4
%942
%MCMLXXXIII

%Sample Output

%25
%IV
%CMXLII
%1983
%http://uva.onlinejudge.org/index.php?option=com_onlinejudge&Itemid=8&category=121&page=show_problem&problem=2663
%=======================================================================================
%=======================================================================================

%Problem code: ONP 
%Transform the algebraic expression with brackets into RPN form (Reverse Polish Notation).
 %Two argument operators: $+$, $-$, $*$, $/$, $^$ 
 %(priority from the lowest to the highest), brackets $( )$. 
 %Operands: only letters: a,b,...,z. Assume that there is only one RPN form (no expressions like a*b*c).
%Input
%t [the number of expressions <= 100]
%expression [length <= 400]
%[other expressions]
%Text grouped in [ ] does not appear in the input file.
%Output
%The expressions in RPN form, one per line.
%Example
%Input:
%\[
%3
%(a+(b*c))
%((a+b)*(z+x))
%((a+t)*((b+(a+c))^(c+d)))
%Output:
%abc*+
%ab+zx+*
%at+bac++cd+^*
%\]
%http://www.spoj.pl
%Added by: Michał Małafiejski
%Date: 2004-05-01
%=======================================================================================
%=======================================================================================
%=======================================================================================
%=======================================================================================

%Problem code: SBSTR1
%Given two binary strings, A (of length 10) and B (of length 5), output 1 if B is a substring of A and 0 
%otherwise.
%Please note, that the solution may only be submitted in the following languages: Brainf**k,
%Whitespace and Intercal. 
%Input
%24 lines consisting of pairs of binary strings A and B separated by a single space.
%Output
%The logical value of: ’B is a substring of A’.
%Example
%First two lines of input:
%1010110010 10110
%1110111011 10011
%First two lines of output:
%1
%0
%http://www.spoj.pl
%Added by: Adrian Kosowski
%Date: 2004-05-01



