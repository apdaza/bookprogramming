\chapter{Estructuras de Iteraci'on}\label{cap:iteracion}
\section{Iteraci�n}
%=======================================================================================
%=======================================================================================
\paragraph*{\ejercicio}  \par %ejercicio035
Escribir un programa que visualice en pantalla los n�meros m�ltiplos de 5 comprendidos entre 1 y 100.

%Lenguaje C
\lstinputlisting[language=C, caption={Lenguaje C $\Longleftarrow$ Ejercicio \theejercicio}, label={\theejercicio b}]{ejercicios/c++/ejercicio035.cpp}

%lenguaje python
\lstinputlisting[language=Python, caption={Lenguaje Python $\Longleftarrow$ Ejercicio \theejercicio}, label={\theejercicio d}]{ejercicios/python/ejercicio035.py}


%=======================================================================================
%=======================================================================================

%=======================================================================================
%=======================================================================================
\paragraph*{\ejercicio}  \par %ejercicio036
Escribir un programa que capture dos n�meros enteros y muestre en pantalla los n�meros comprendidos entre estos dos n�meros\\

%Lenguaje C
\lstinputlisting[language=C, caption={Lenguaje C $\Longleftarrow$ Ejercicio \theejercicio}, label={\theejercicio b}]{ejercicios/c++/ejercicio036.cpp}

%lenguaje python
\lstinputlisting[language=Python, caption={Lenguaje Python $\Longleftarrow$ Ejercicio \theejercicio}, label={\theejercicio d}]{ejercicios/python/ejercicio036.py}

%=======================================================================================
%=======================================================================================

%=======================================================================================
%=======================================================================================
\paragraph*{\ejercicio}  \par %ejercicio037
Escribir un programa que capture dos n�meros enteros y muestre en pantalla los n�meros pares comprendidos entre estos dos n�meros.\\

%Lenguaje C
\lstinputlisting[language=C, caption={Lenguaje C $\Longleftarrow$ Ejercicio \theejercicio}, label={\theejercicio b}]{ejercicios/c++/ejercicio037.cpp}

%lenguaje python
\lstinputlisting[language=Python, caption={Lenguaje Python $\Longleftarrow$ Ejercicio \theejercicio}, label={\theejercicio d}]{ejercicios/python/ejercicio037.py}
%=======================================================================================
%=======================================================================================


%=======================================================================================
%=======================================================================================
\paragraph*{\ejercicio}  \par %Ejercicio 038
Escribir un programa que genere la tabla de multiplicar de un n�mero introducido por el teclado.\\

%Lenguaje C
\lstinputlisting[language=C, caption={Lenguaje C $\Longleftarrow$ Ejercicio \theejercicio}, label={\theejercicio b}]{ejercicios/c++/ejercicio038.cpp}

%lenguaje python
\lstinputlisting[language=Python, caption={Lenguaje Python $\Longleftarrow$ Ejercicio \theejercicio}, label={\theejercicio d}]{ejercicios/python/ejercicio038.py}

%=======================================================================================
%=======================================================================================

%=======================================================================================
%=======================================================================================
\paragraph*{\ejercicio}  \par %ejercicio 39
Escribir un programa que calcule la media de $5$ n�meros introducidos por el teclado.\\

%Lenguaje C
\lstinputlisting[language=C, caption={Lenguaje C $\Longleftarrow$ Ejercicio \theejercicio}, label={\theejercicio b}]{ejercicios/c++/ejercicio039.cpp}

%lenguaje python
\lstinputlisting[language=Python, caption={Lenguaje Python $\Longleftarrow$ Ejercicio \theejercicio}, label={\theejercicio d}]{ejercicios/python/ejercicio039.py}

%=======================================================================================
%=======================================================================================

%=======================================================================================
%=======================================================================================
\paragraph*{\ejercicio}  \par %ejercicio 40
Escribir un programa que realice la pregunta desea continuar S/N? y que no deje de hacerla hasta que el usuario teclee N y muestre en pantalla la cantidad de veces que se realizo la pregunta. \\

%Lenguaje C
\lstinputlisting[language=C, caption={Lenguaje C $\Longleftarrow$ Ejercicio \theejercicio}, label={\theejercicio b}]{ejercicios/c++/ejercicio040.cpp}

%lenguaje python
\lstinputlisting[language=Python, caption={Lenguaje Python $\Longleftarrow$ Ejercicio \theejercicio}, label={\theejercicio d}]{ejercicios/python/ejercicio040.py}
%=======================================================================================
%=======================================================================================

%=======================================================================================
%=======================================================================================
\paragraph*{\ejercicio}  \par %Ejercicio 041
Escribir un programa que escriba los n�meros comprendidos entre 1 y 1000. El programa escribir� en la pantalla los n�meros en grupos de 20, solicitando al usuario si quiere o no continuar visualizando el siguiente grupo de n�meros.\\

%Lenguaje C
\lstinputlisting[language=C, caption={Lenguaje C $\Longleftarrow$ Ejercicio \theejercicio}, label={\theejercicio b}]{ejercicios/c++/ejercicio041.cpp}

%lenguaje python
\lstinputlisting[language=Python, caption={Lenguaje Python $\Longleftarrow$ Ejercicio \theejercicio}, label={\theejercicio d}]{ejercicios/python/ejercicio041.py}

%=======================================================================================
%=======================================================================================


%=======================================================================================
%=======================================================================================
\paragraph*{\ejercicio}  \par %Ejercicio 042
Escribir un programa que calcule la media de n�meros introducidos por el teclado hasta que el n�mero ingresado sea cero.\\

%Lenguaje C
\lstinputlisting[language=C, caption={Lenguaje C $\Longleftarrow$ Ejercicio \theejercicio}, label={\theejercicio b}]{ejercicios/c++/ejercicio042.cpp}

%lenguaje python
\lstinputlisting[language=Python, caption={Lenguaje Python $\Longleftarrow$ Ejercicio \theejercicio}, label={\theejercicio d}]{ejercicios/python/ejercicio042.py}
%=======================================================================================
%=======================================================================================

%=======================================================================================
%=======================================================================================
\paragraph*{\ejercicio}  \par % Ejercicio 043
Escribir un programa que calcule, independientemente, la suma y la media de los n�meros pares e impares comprendidos entre 1 y 200.\\

%Lenguaje C
\lstinputlisting[language=C, caption={Lenguaje C $\Longleftarrow$ Ejercicio \theejercicio}, label={\theejercicio b}]{ejercicios/c++/ejercicio043.cpp}

%lenguaje python
\lstinputlisting[language=Python, caption={Lenguaje Python $\Longleftarrow$ Ejercicio \theejercicio}, label={\theejercicio d}]{ejercicios/python/ejercicio043.py}
%=======================================================================================
%=======================================================================================

%=======================================================================================
%=======================================================================================
\paragraph*{\ejercicio}  \par %Ejercicio 044
Escribir un programa que calcule la suma de los cuadrados de los 100 primeros n�meros enteros.\\

%Lenguaje C
\lstinputlisting[language=C, caption={Lenguaje C $\Longleftarrow$ Ejercicio \theejercicio}, label={\theejercicio b}]{ejercicios/c++/ejercicio044.cpp}

%lenguaje python
\lstinputlisting[language=Python, caption={Lenguaje Python $\Longleftarrow$ Ejercicio \theejercicio}, label={\theejercicio d}]{ejercicios/python/ejercicio044.py}

%=======================================================================================
%=======================================================================================


%=======================================================================================
%=======================================================================================
\paragraph*{\ejercicio} \par %ejercicio 45
Leer un n�mero entero y determinar si es un n�mero primo. \\

\shadowbox{
\begin{minipage}{4in}
\textcolor{blue}{{\bf N�mero primo:} Un n�mero $n$ es primo si es divisible �nicamente por la unidad y por el mismo.} \\
\end{minipage}
}

%Lenguaje C
\lstinputlisting[language=C, caption={Lenguaje C $\Longleftarrow$ Ejercicio \theejercicio}, label={\theejercicio b}]{ejercicios/c++/ejercicio045.cpp}

%lenguaje python
\lstinputlisting[language=Python, caption={Lenguaje Python $\Longleftarrow$ Ejercicio \theejercicio}, label={\theejercicio d}]{ejercicios/python/ejercicio045.py}


%=======================================================================================
%=======================================================================================


%=======================================================================================
%=======================================================================================
\paragraph*{\ejercicio}  \par %Ejercicio 046
Imprimir los n�meros perfectos comprendidos en un intervalo dado.\\

\shadowbox{
\begin{minipage}{4in}
\textcolor{blue}{{\bf N�mero perfecto:} Un n�mero $n$ es perfecto si la suma de los divisores propios del n�mero es igual a $n$.}
\end{minipage}
}

%Lenguaje C
\lstinputlisting[language=C, caption={Lenguaje C $\Longleftarrow$ Ejercicio \theejercicio}, label={\theejercicio b}]{ejercicios/c++/ejercicio046.cpp}

%lenguaje python
\lstinputlisting[language=Python, caption={Lenguaje Python $\Longleftarrow$ Ejercicio \theejercicio}, label={\theejercicio d}]{ejercicios/python/ejercicio046.py}
%=======================================================================================
%=======================================================================================

%=======================================================================================
%=======================================================================================
\paragraph*{\ejercicio}  \par %Ejercicio 47
Juego Adivinar n�mero entre 0 y n, con k intentos.\\

%Lenguaje C
\lstinputlisting[language=C, caption={Lenguaje C $\Longleftarrow$ Ejercicio \theejercicio}, label={\theejercicio b}]{ejercicios/c++/ejercicio047.cpp}


%lenguaje python
\lstinputlisting[language=Python, caption={Lenguaje Python $\Longleftarrow$ Ejercicio \theejercicio}, label={\theejercicio d}]{ejercicios/python/ejercicio047.py}
%=======================================================================================
%=======================================================================================


%=======================================================================================
%=======================================================================================
\paragraph*{\ejercicio}  \par %Ejercicio 48
Escribir un programa que genere la serie de fibonacci hasta el indice que desee el usuario 

%Lenguaje C
\lstinputlisting[language=C, caption={Lenguaje C $\Longleftarrow$ Ejercicio \theejercicio}, label={\theejercicio b}]{ejercicios/c++/ejercicio048.cpp}


%lenguaje python
\lstinputlisting[language=Python, caption={Lenguaje Python $\Longleftarrow$ Ejercicio \theejercicio}, label={\theejercicio d}]{ejercicios/python/ejercicio048.py}
%=======================================================================================
%=======================================================================================

%=======================================================================================
%=======================================================================================
\paragraph*{\ejercicio}  \par %Ejercicio 49
Escribir un programa que permita multiplicar dos n�meros leidos desde el teclado sin usar el operador de multiplicaci�n

%Lenguaje C
\lstinputlisting[language=C, caption={Lenguaje C $\Longleftarrow$ Ejercicio \theejercicio}, label={\theejercicio b}]{ejercicios/c++/ejercicio049.cpp}


%lenguaje python
\lstinputlisting[language=Python, caption={Lenguaje Python $\Longleftarrow$ Ejercicio \theejercicio}, label={\theejercicio d}]{ejercicios/python/ejercicio049.py}
%=======================================================================================
%=======================================================================================

%=======================================================================================
%=======================================================================================
\paragraph*{\ejercicio}  \par %Ejercicio 50
Escriba un programa que divida dos numeros leidos sin usar el operador de la divisi�n.

%Lenguaje C
\lstinputlisting[language=C, caption={Lenguaje C $\Longleftarrow$ Ejercicio \theejercicio}, label={\theejercicio b}]{ejercicios/c++/ejercicio050.cpp}


%lenguaje python
\lstinputlisting[language=Python, caption={Lenguaje Python $\Longleftarrow$ Ejercicio \theejercicio}, label={\theejercicio d}]{ejercicios/python/ejercicio050.py}
%=======================================================================================
%=======================================================================================

%=======================================================================================
%=======================================================================================
\paragraph*{\ejercicio}  \par %Ejercicio 51
Escribir un programa que genere dos columnas de n�meros, la primera con los n�meros del 1 al 10 y la segunda con los numeros del 10 al 1

%Lenguaje C
\lstinputlisting[language=C, caption={Lenguaje C $\Longleftarrow$ Ejercicio \theejercicio}, label={\theejercicio b}]{ejercicios/c++/ejercicio051.cpp}


%lenguaje python
\lstinputlisting[language=Python, caption={Lenguaje Python $\Longleftarrow$ Ejercicio \theejercicio}, label={\theejercicio d}]{ejercicios/python/ejercicio051.py}
%=======================================================================================
%=======================================================================================

%=======================================================================================
%=======================================================================================
\paragraph*{\ejercicio}  \par %Ejercicio 52
Escribir un programa que lea una serie de n�meros desde el teclado hasta que el n�mero ingresado sea cero y muestre en pantalla el mayor n�mero leido, el menor n�mero le�do y el promedio de los n�meros leidos (no debe tener en cuenta el cero)

%Lenguaje C
\lstinputlisting[language=C, caption={Lenguaje C $\Longleftarrow$ Ejercicio \theejercicio}, label={\theejercicio b}]{ejercicios/c++/ejercicio052.cpp}


%lenguaje python
\lstinputlisting[language=Python, caption={Lenguaje Python $\Longleftarrow$ Ejercicio \theejercicio}, label={\theejercicio d}]{ejercicios/python/ejercicio052.py}
%=======================================================================================
%=======================================================================================

%=======================================================================================
%=======================================================================================
\paragraph*{\ejercicio}  \par %Ejercicio 53
Escribir un programa que calcule el factorial de un n�mero leido por el teclado

%Lenguaje C
\lstinputlisting[language=C, caption={Lenguaje C $\Longleftarrow$ Ejercicio \theejercicio}, label={\theejercicio b}]{ejercicios/c++/ejercicio053.cpp}


%lenguaje python
\lstinputlisting[language=Python, caption={Lenguaje Python $\Longleftarrow$ Ejercicio \theejercicio}, label={\theejercicio d}]{ejercicios/python/ejercicio053.py}
%=======================================================================================
%=======================================================================================

%=======================================================================================
%=======================================================================================
\paragraph*{\ejercicio}  \par %Ejercicio 54
Escribir un programa que lea un n�mero entero y lo descomponga en sus d�gitos, mostrando primero las unidades, luego las decenas, centenas y as� hasta terminar los digitos


%Lenguaje C
\lstinputlisting[language=C, caption={Lenguaje C $\Longleftarrow$ Ejercicio \theejercicio}, label={\theejercicio b}]{ejercicios/c++/ejercicio054.cpp}


%lenguaje python
\lstinputlisting[language=Python, caption={Lenguaje Python $\Longleftarrow$ Ejercicio \theejercicio}, label={\theejercicio d}]{ejercicios/python/ejercicio054.py}
%=======================================================================================
%=======================================================================================


















