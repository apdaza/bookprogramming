\chapter{Variables y Operadores}\label{cap:variables}

\section{Variables}
%=======================================================================================
%=======================================================================================
\paragraph*{\ejercicio}  \par %ejercicio001
Analizar, dise�ar en implementar un programa de computador que pregunte al usuario el nombre, e imprima en pantalla un saludo de bienvenida acompa�ado del nombre que fue ingresado.\\


%Lenguaje C
\lstinputlisting[language=C, caption={Lenguaje C $\Longleftarrow$ Ejercicio \theejercicio}, label={\theejercicio b}]{ejercicios/c++/ejercicio001.cpp}


%lenguaje python
\lstinputlisting[language=Python, caption={Lenguaje Pyton $\Longleftarrow$ Ejercicio \theejercicio}, label={\theejercicio d}]{ejercicios/python/ejercicio001.py}

%=======================================================================================
%=======================================================================================


%=======================================================================================
%=======================================================================================
\paragraph*{\ejercicio}  \par %ejercicio002
Escribir un programa que calcule la hipotenusa de un tri�ngulo rect�ngulo capturando el valor de los catetos (opuesto y adyacente). \\


%Lenguaje C
\lstinputlisting[language=C, caption={Lenguaje C $\Longleftarrow$ Ejercicio \theejercicio}, label={\theejercicio b}]{ejercicios/c++/ejercicio002.cpp}

%lenguaje python
\lstinputlisting[language=Python, caption={Lenguaje Pyton $\Longleftarrow$ Ejercicio \theejercicio}, label={\theejercicio d}]{ejercicios/python/ejercicio002.py}

%=======================================================================================
%=======================================================================================	


%=======================================================================================
%=======================================================================================
\paragraph*{\ejercicio}  \par %ejercicio003
Escribir un programa que calcule el n�mero de horas, minutos y segundos que hay en un valor x de segundos indicados por el usuario. \\

%Lenguaje C
\lstinputlisting[language=C, caption={Lenguaje C $\Longleftarrow$ Ejercicio \theejercicio}, label={\theejercicio b}]{ejercicios/c++/ejercicio003.cpp}

%lenguaje python
\lstinputlisting[language=Python, caption={Lenguaje Pyton $\Longleftarrow$ Ejercicio \theejercicio}, label={\theejercicio d}]{ejercicios/python/ejercicio003.py}
%=======================================================================================
%=======================================================================================	

%=======================================================================================
%=======================================================================================
\paragraph*{\ejercicio}  \par %ejercicio004
Escribir un programa que calcule el equivalente en pies de una longitud de x metros sabiendo que 1 metro equivale a 39.27 pulgadas y 12 pulgadas equivalen a 1 pie. 


%Lenguaje C
\lstinputlisting[language=C, caption={Lenguaje C $\Longleftarrow$ Ejercicio \theejercicio}, label={\theejercicio b}]{ejercicios/c++/ejercicio004.cpp}

%lenguaje python
\lstinputlisting[language=Python, caption={Lenguaje Pyton $\Longleftarrow$ Ejercicio \theejercicio}, label={\theejercicio d}]{ejercicios/python/ejercicio004.py}
%=======================================================================================
%=======================================================================================	

%=======================================================================================
%=======================================================================================
\paragraph*{\ejercicio}  \par %ejercicio005
Escribir un programa que calcule la distancia entre dos puntos a partir de sus coordenadas (x1, y1), (x2, y2).

%Lenguaje C
\lstinputlisting[language=C, caption={Lenguaje C $\Longleftarrow$ Ejercicio \theejercicio}, label={\theejercicio b}]{ejercicios/c++/ejercicio005.cpp}

%lenguaje python
\lstinputlisting[language=Python, caption={Lenguaje Pyton $\Longleftarrow$ Ejercicio \theejercicio}, label={\theejercicio d}]{ejercicios/python/ejercicio005.py}
%=======================================================================================
%=======================================================================================	

%=======================================================================================
%=======================================================================================
\paragraph*{\ejercicio}  \par %ejercicio006
Escribir un programa que lea dos n�meros enteros positivos mayores que 1,  A y B, y obtenga los valores de la divisi�n entera y el residuo  de A dividido B.

%Lenguaje C
\lstinputlisting[language=C, caption={Lenguaje C $\Longleftarrow$ Ejercicio \theejercicio}, label={\theejercicio b}]{ejercicios/c++/ejercicio006.cpp}

%lenguaje python
\lstinputlisting[language=Python, caption={Lenguaje Python $\Longleftarrow$ Ejercicio \theejercicio}, label={\theejercicio d}]{ejercicios/python/ejercicio006.py}


%=======================================================================================
%=======================================================================================	

%=======================================================================================
%=======================================================================================
\paragraph*{\ejercicio}  \par %ejercicio 007
Escribir un programa que calcule la velocidad de un proyectil que recorre x Km en t minutos, expresando el resultado en metros/segundo (Velocidad = espacio/tiempo)

%Lenguaje C
\lstinputlisting[language=C, caption={Lenguaje C $\Longleftarrow$ Ejercicio \theejercicio}, label={\theejercicio b}]{ejercicios/c++/ejercicio007.cpp}

%lenguaje python
\lstinputlisting[language=Python, caption={Lenguaje Python $\Longleftarrow$ Ejercicio \theejercicio}, label={\theejercicio d}]{ejercicios/python/ejercicio007.py}
%=======================================================================================
%=======================================================================================	

%=======================================================================================
%=======================================================================================
\paragraph*{\ejercicio}  \par %ejercicio 008
Escribir un programa que lea dos n�meros enteros A y B, y obtenga los valores de la divisi�n entera de A dividido B y el residuo de esta sin hacer uso del operador m�dulo.

%Lenguaje C
\lstinputlisting[language=C, caption={Lenguaje C $\Longleftarrow$ Ejercicio \theejercicio}, label={\theejercicio b}]{ejercicios/c++/ejercicio008.cpp}

%lenguaje python
\lstinputlisting[language=Python, caption={Lenguaje Python $\Longleftarrow$ Ejercicio \theejercicio}, label={\theejercicio d}]{ejercicios/python/ejercicio008.py}

%=======================================================================================
%=======================================================================================	



\section{Operadores}
%=======================================================================================
%=======================================================================================
\paragraph*{\ejercicio}  \par %Ejercicio 9
Escribir un programa que muestre el resultado de sumar, restar, multiplicar y dividir dos n�meros enteros mayores que cero le�dos desde el teclado.\\

%Lenguaje C
\lstinputlisting[language=C, caption={Lenguaje C $\Longleftarrow$ Ejercicio \theejercicio}, label={\theejercicio b}]{ejercicios/c++/ejercicio009.cpp}

%lenguaje python
\lstinputlisting[language=Python, caption={Lenguaje Python $\Longleftarrow$ Ejercicio \theejercicio}, label={\theejercicio d}]{ejercicios/python/ejercicio009.py}


%=======================================================================================
%=======================================================================================	

%=======================================================================================
%=======================================================================================
\paragraph*{\ejercicio}  \par %ejercicio10
Escribir un programa que lea dos n�meros enteros, los almacene en dos variables e  intercambie su contenido. \\

%Lenguaje C
\lstinputlisting[language=C, caption={Lenguaje C $\Longleftarrow$ Ejercicio \theejercicio}, label={\theejercicio b}]{ejercicios/c++/ejercicio010.cpp}

%lenguaje python
\lstinputlisting[language=Python, caption={Lenguaje Python $\Longleftarrow$ Ejercicio \theejercicio}, label={\theejercicio d}]{ejercicios/python/ejercicio010.py}

%=======================================================================================
%=======================================================================================	

%=======================================================================================
%=======================================================================================
\paragraph*{\ejercicio}  \par %ejercicio11
Escribir un programa que, dado un n�mero total de horas, devuelve el n�mero de semanas, d�as y horas equivalentes. \\

%Lenguaje C
\lstinputlisting[language=C, caption={Lenguaje C $\Longleftarrow$ Ejercicio \theejercicio}, label={\theejercicio b}]{ejercicios/c++/ejercicio011.cpp}

%lenguaje python
\lstinputlisting[language=Python, caption={Lenguaje Python $\Longleftarrow$ Ejercicio \theejercicio}, label={\theejercicio d}]{ejercicios/python/ejercicio011.py}

%=======================================================================================
%=======================================================================================	


%=======================================================================================
%=======================================================================================
\paragraph*{\ejercicio}  \par %ejercicio12
Escribir un programa que dado un n�mero muestre en pantalla el doble del n�mero y la mitad del n�mero sin usar los operadores de divisi�n ni multiplicaci�n

%Lenguaje C
\lstinputlisting[language=C, caption={Lenguaje C $\Longleftarrow$ Ejercicio \theejercicio}, label={\theejercicio b}]{ejercicios/c++/ejercicio012.cpp}

%lenguaje python
\lstinputlisting[language=Python, caption={Lenguaje Python $\Longleftarrow$ Ejercicio \theejercicio}, label={\theejercicio d}]{ejercicios/python/ejercicio012.py}
%=======================================================================================
%=======================================================================================	


%=======================================================================================
%=======================================================================================
\paragraph*{\ejercicio}  \par %ejercicio13
Escribir un programa que genere un n�mero aleatorio entre 1 y 99 y muestre el resto de la divisi�n entera por 2, 3, 5 y 7

%Lenguaje C
\lstinputlisting[language=C, caption={Lenguaje C $\Longleftarrow$ Ejercicio \theejercicio}, label={\theejercicio b}]{ejercicios/c++/ejercicio013.cpp}

%lenguaje python
\lstinputlisting[language=Python, caption={Lenguaje Python $\Longleftarrow$ Ejercicio \theejercicio}, label={\theejercicio d}]{ejercicios/python/ejercicio013.py}

%=======================================================================================
%=======================================================================================	

%=======================================================================================
%=======================================================================================
\paragraph*{\ejercicio}  \par %ejercicio14
Escribir un programa que lea dos n�meros, calcule la potencia del primero elevado al segundo y muestre el resultado en pantalla

%Lenguaje C
\lstinputlisting[language=C, caption={Lenguaje C $\Longleftarrow$ Ejercicio \theejercicio}, label={\theejercicio b}]{ejercicios/c++/ejercicio014.cpp}

%lenguaje python
\lstinputlisting[language=Python, caption={Lenguaje Python $\Longleftarrow$ Ejercicio \theejercicio}, label={\theejercicio d}]{ejercicios/python/ejercicio014.py}
%=======================================================================================
%=======================================================================================	


%=======================================================================================
%=======================================================================================
\paragraph*{\ejercicio}  \par %ejercicio15
Escribir un programa que invierta bit a bit un n�mero ingresado por teclado, ejemplo: si el n�mero ingresado es 5 este se representa de forma binaria en 00101 y el resultado de negar bit a bit este n�mero es 11010 que coresponde a -6

%Lenguaje C
\lstinputlisting[language=C, caption={Lenguaje C $\Longleftarrow$ Ejercicio \theejercicio}, label={\theejercicio b}]{ejercicios/c++/ejercicio015.cpp}

%lenguaje python
\lstinputlisting[language=Python, caption={Lenguaje Python $\Longleftarrow$ Ejercicio \theejercicio}, label={\theejercicio d}]{ejercicios/python/ejercicio015.py}
%=======================================================================================
%=======================================================================================	


%=======================================================================================
%=======================================================================================
\paragraph*{\ejercicio}  \par %ejercicio16
Escribir un programa que calcule el resultado de duplicar acumulativamente un n�mero leido por teclado la cantidad de veces indicadas por el usuario

%Lenguaje C
\lstinputlisting[language=C, caption={Lenguaje C $\Longleftarrow$ Ejercicio \theejercicio}, label={\theejercicio b}]{ejercicios/c++/ejercicio016.cpp}

%lenguaje python
\lstinputlisting[language=Python, caption={Lenguaje Python $\Longleftarrow$ Ejercicio \theejercicio}, label={\theejercicio d}]{ejercicios/python/ejercicio016.py}

%=======================================================================================
%=======================================================================================	


%=======================================================================================
%=======================================================================================
\paragraph*{\ejercicio}  \par %ejercicio17
Escribir un programa que calcule el valor de sumar dos enteros sin tener en cuenta el arrastre a nivel binario en la suma

%usar el xor
%Lenguaje C
\lstinputlisting[language=C, caption={Lenguaje C $\Longleftarrow$ Ejercicio \theejercicio}, label={\theejercicio b}]{ejercicios/c++/ejercicio017.cpp}


%lenguaje python
\lstinputlisting[language=Python, caption={Lenguaje Python $\Longleftarrow$ Ejercicio \theejercicio}, label={\theejercicio d}]{ejercicios/python/ejercicio017.py}
%=======================================================================================
%=======================================================================================	


%=======================================================================================
%=======================================================================================
\paragraph*{\ejercicio}  \par %ejercicio18
Escxribir un programa que calcule el residuo de la divisi�n entera de dos n�meros leidos por teclado

%Lenguaje C
\lstinputlisting[language=C, caption={Lenguaje C $\Longleftarrow$ Ejercicio \theejercicio}, label={\theejercicio b}]{ejercicios/c++/ejercicio018.cpp}

%lenguaje python
\lstinputlisting[language=Python, caption={Lenguaje Python $\Longleftarrow$ Ejercicio \theejercicio}, label={\theejercicio d}]{ejercicios/python/ejercicio018.py}
%=======================================================================================
%=======================================================================================	


%=======================================================================================
%=======================================================================================
\paragraph*{\ejercicio}  \par %ejercicio19
Escribir un programa que lea el radio de un c�rculo como un n�mero flotante y muestre el �rea y el per�metro del c�rculo.

%Lenguaje C
\lstinputlisting[language=C, caption={Lenguaje C $\Longleftarrow$ Ejercicio \theejercicio}, label={\theejercicio b}]{ejercicios/c++/ejercicio019.cpp}

%lenguaje python
\lstinputlisting[language=Python, caption={Lenguaje Python $\Longleftarrow$ Ejercicio \theejercicio}, label={\theejercicio d}]{ejercicios/python/ejercicio019.py}
%=======================================================================================
%=======================================================================================	







